\documentclass[]{article}
\usepackage{lmodern}
\usepackage{amssymb,amsmath}
\usepackage{ifxetex,ifluatex}
\usepackage{fixltx2e} % provides \textsubscript
\ifnum 0\ifxetex 1\fi\ifluatex 1\fi=0 % if pdftex
  \usepackage[T1]{fontenc}
  \usepackage[utf8]{inputenc}
\else % if luatex or xelatex
  \ifxetex
    \usepackage{mathspec}
  \else
    \usepackage{fontspec}
  \fi
  \defaultfontfeatures{Ligatures=TeX,Scale=MatchLowercase}
\fi
% use upquote if available, for straight quotes in verbatim environments
\IfFileExists{upquote.sty}{\usepackage{upquote}}{}
% use microtype if available
\IfFileExists{microtype.sty}{%
\usepackage{microtype}
\UseMicrotypeSet[protrusion]{basicmath} % disable protrusion for tt fonts
}{}
\usepackage[margin=1in]{geometry}
\usepackage{hyperref}
\hypersetup{unicode=true,
            pdftitle={Rounding in the age of statcheck},
            pdfauthor={Felix Thoemmes},
            pdfborder={0 0 0},
            breaklinks=true}
\urlstyle{same}  % don't use monospace font for urls
\usepackage{longtable,booktabs}
\usepackage{graphicx,grffile}
\makeatletter
\def\maxwidth{\ifdim\Gin@nat@width>\linewidth\linewidth\else\Gin@nat@width\fi}
\def\maxheight{\ifdim\Gin@nat@height>\textheight\textheight\else\Gin@nat@height\fi}
\makeatother
% Scale images if necessary, so that they will not overflow the page
% margins by default, and it is still possible to overwrite the defaults
% using explicit options in \includegraphics[width, height, ...]{}
\setkeys{Gin}{width=\maxwidth,height=\maxheight,keepaspectratio}
\IfFileExists{parskip.sty}{%
\usepackage{parskip}
}{% else
\setlength{\parindent}{0pt}
\setlength{\parskip}{6pt plus 2pt minus 1pt}
}
\setlength{\emergencystretch}{3em}  % prevent overfull lines
\providecommand{\tightlist}{%
  \setlength{\itemsep}{0pt}\setlength{\parskip}{0pt}}
\setcounter{secnumdepth}{0}
% Redefines (sub)paragraphs to behave more like sections
\ifx\paragraph\undefined\else
\let\oldparagraph\paragraph
\renewcommand{\paragraph}[1]{\oldparagraph{#1}\mbox{}}
\fi
\ifx\subparagraph\undefined\else
\let\oldsubparagraph\subparagraph
\renewcommand{\subparagraph}[1]{\oldsubparagraph{#1}\mbox{}}
\fi

%%% Use protect on footnotes to avoid problems with footnotes in titles
\let\rmarkdownfootnote\footnote%
\def\footnote{\protect\rmarkdownfootnote}

%%% Change title format to be more compact
\usepackage{titling}

% Create subtitle command for use in maketitle
\newcommand{\subtitle}[1]{
  \posttitle{
    \begin{center}\large#1\end{center}
    }
}

\setlength{\droptitle}{-2em}
  \title{Rounding in the age of statcheck}
  \pretitle{\vspace{\droptitle}\centering\huge}
  \posttitle{\par}
  \author{Felix Thoemmes}
  \preauthor{\centering\large\emph}
  \postauthor{\par}
  \predate{\centering\large\emph}
  \postdate{\par}
  \date{2017-06-05}


\begin{document}
\maketitle

I recently chatted with a colleague whose work was flagged on PubPeer by
the statcheck bot that has been scraping the web last year or so. Her
reaction - which I believe is pretty typical - was mild annoyance. To
her (and I believe many others) it felt a bit pedantic to be flagged
what was essentially a rounding error. She feared that such a flag may
create a false impression that the work is a ``false positive'', and
that people jump to conclusions when they see a statcheck error.

I won't be writing about this aspect of statcheck though. My colleague
did make one point that I thought was worth some investigation. She
mentioned that she rounded both the reported t-value and the p-value,
and that statcheck (when recomputing the p-value from the rounded
t-value) would therefore make a mistake, and falsely label her results
as incorrect. Let's unpack this a bit\ldots{}

\subsubsection{The potential problem of
rounding}\label{the-potential-problem-of-rounding}

Assume you have obtained a t-value of 3.454 (with 15 degrees of
freedom). Your computed p-value would be 0.0035 (which rounds to .004).
But the t-value itself could be reported in a rounded format in the
paper. You could report 3.454 (with three digits), 3.45 (round to two
digits), or you could even report 3.5 (round to one digit, unusual, but
not unseen). For the last two rounded values, statcheck would assume
that the t-value is 3.450, and 3.500 (contrary to the fact that it is
3.454), and compute a p-value from those values. That recomputed p-value
would then not match the reported one, and an error would be incorrectly
flagged. Here is an example in a table.

\begin{longtable}[]{@{}llllll@{}}
\toprule
t & p & rounded t & rounded p & statcheck t & statcheck p\tabularnewline
\midrule
\endhead
3.454 & 0.0035 & 3.45 & .004 & 3.450 & 0.004\tabularnewline
3.454 & 0.0035 & 3.5 & .004 & 3.500 & 0.003\tabularnewline
\bottomrule
\end{longtable}

Note the second column in which the researcher would correctly report
.004, but statcheck would recompute .003, and thus flag an error.

\subsubsection{Actually, statcheck makes no such
error}\label{actually-statcheck-makes-no-such-error}

After my colleague told me this argument, I went ahead and tested this
in statcheck\ldots{} and it turns out it makes no such mistake. I probed
statcheck a little bit and then looked at the sourcecode. Statchec will
report the p-value that it obtains from the rounded t-value (what other
p-value could it report, really), and while it is also true that the
reported and the recomputed p-value will be different, statcheck will
only report an error when the difference between the two p-values is
inconsitent with rounding based on the reported digits of the t-value.
In other words, statcheck checks the accuracy with which you reported
your t-value (i.e., the number of decimal places), and then it is smart
enough to realize whether you made a rounding error, or whether any
discrepancy is due to innocent rounding that is in line with whatever
number of digits you rounded to. So what's the bottom line here: you can
safley assume that statcheck will not flag you for rounding. It does its
job as it's supposed to do. And that should be a comforting thought.


\end{document}
